\documentclass{article}   % Specifies the document class
\usepackage[bahasa]{babel}
\usepackage{lipsum}
\usepackage{amsmath}  % For math
\usepackage{amssymb}  % For more math
\usepackage{enumerate} % useful for itemization
\usepackage{enumitem}
\usepackage{siunitx}  % standardization of si units
\usepackage{graphicx}
%Stop indentation on new paragraphs
\usepackage[parfill]{parskip}
\usepackage{lastpage}
\usepackage{setspace}
\usepackage{geometry}
\geometry{
	a4paper,
	%total={170mm,257mm},
	left=20mm,
	right=20mm,
	top=25mm,
	bottom=20mm,
}
% Lengkapi Isian Berikut
\newcommand{\judul}{Berikan Judul}
\newcommand{\nama}{Nama anda tanpa gelar}
\newcommand{\gelardepan}{Dr.}
\newcommand{\gelarbelakang}{S.Si.,M.Si.}
\newcommand{\email}{an.email@gmail.com}
\newcommand{\judulfooter}{\emph{judul untuk footers}}
\newcommand{\matkul}{Topik}
\newcommand{\versi}{2023 v1.01}
\newcommand{\asosiasi}{Asosiasi Anda}

\usepackage{amsmath}  % For math
\usepackage{amssymb}  % For more math
\usepackage{enumerate} % useful for itemization
\usepackage{enumitem}
\usepackage{siunitx}  % standardization of si units
\usepackage{graphicx}
\graphicspath{{./gambar/}} %picture path
\usepackage{pgfplots}
\pgfplotsset{compat = newest}
%Stop indentation on new paragraphs
%\usepackage[parfill]{parskip}
\usepackage{lastpage}
\usepackage{setspace}
\usepackage{xcolor}

%============== For header and footer
\usepackage{fancyhdr}
\pagestyle{fancy}
%  --------   Normal Headers
\fancyhf{} % clear all fields
\fancyhead[L]{\asosiasi}
\fancyfoot[L]{\nama, ~\judulfooter}
%\fancyfoot[C]{}
\fancyfoot[R]{\textsf{Tulisan \versi~pages \textsf{\thepage~of~\pageref{LastPage}}}}
\renewcommand{\headrulewidth}{0pt} % to remove line on header
\renewcommand{\footrulewidth}{0pt} % to remove line on footer
%==============================++==============================
\usepackage[numbers]{natbib}
% Since natbib uses an unnumbered section by default we define \bibsection
% to make it a regular numbered section.
\renewcommand \bibsection{\section{Daftar Pustaka}}
 %citecolor
 \usepackage[colorlinks,linkcolor={black},citecolor={blue!80!black},urlcolor={blue!80!black}]{hyperref}
  \bibliographystyle{ieeetr}
% %==============

\begin{document}

\thispagestyle{empty}
\begin{center}
    \vspace{.4cm}
    \textsf{\textbf { \large \judul}}
\end{center}
\vspace{.4cm}
\hrule
	\textsf{\\
		\textbf{\gelardepan\nama,\gelarbelakang} \hspace{\fill}
		\textit{\large }\\ [0.7ex]
		\textit{\email} \hspace{\fill} \textbf{}\\ [0.7ex]
		\asosiasi \hspace{\fill} \textbf{- \matkul }} \\
	\hrule
\vspace{.4cm}
%============================MULAI DOKUMEN
\begin{center}
\textbf{\Large Motivasi}
\end{center}
\paragraph*{} 
\lipsum[4]\cite{zz}

%section
		\section{Materi dasar 1}
		\paragraph*{} \lipsum[3]\cite{zz1}
		\begin{align*}
		A = \left[
		\begin{array}{rr} 
			2 & 1 \\
			3 & -1 
		\end{array} 
		\right]
		\end{align*}
		\lipsum[7]
		\subsection[]{Materi dasar 1.1} 
		\lipsum[2]
		$$K_{ij} = (-1)^{i+j}M_{ij}$$
		\begin{align*}
			K_{22} &= (-1)^{4}M_{22}\\
			K_{22} &= 5
		\end{align*}
		\subsection{Materi dasar 1.2}
		% Transformasi ini tidak membatasi ordo matriks, sehingga untuk ordo berapapun dapat diselesaikan. Tambahan bahwa untuk ukuran matriks yang lebih besar, dapat diterapkan dua operasi sekaligus agar langkah proses lebih efisien. Misalkan matriks ordo $3\times3$ berikut
				\begin{align*}
		\left[
		\begin{array}{rrr} 
			-2 & 1 & 2\\
			2 & 3 & 1\\
			-2 &1 & 3 
		\end{array} 
		\right]
	\end{align*}
	\lipsum[6]
\section{Materi Dasar 2}
		\paragraph*{} \lipsum[3]\cite{zz2}
		\subsection{Materi 2.1}
		\lipsum[9]
\subsection{Materi 2.2}
\lipsum[2]
\section{Materi Dasar 3}
		\paragraph*{} \lipsum[7]\cite{zz3}
		$$2x_1 + x_2  = 4$$
		$$3x_1 - x_2  = 1$$ 
		\paragraph*{} \lipsum[8]\cite{zz1,zz3}

\section{Kesimpulan}
\paragraph*{} \lipsum[5]\cite{zz,zz1,zz2,zz3}
\section{Tugas Mandiri}
		\begin{enumerate}
		\item Ubah matriks berikut menjadi matriks identitas menggunakan transformasi matriks
		$$
		\begin{array}{llllllllll} 
			a. & 	A=\left[
			\begin{array}{rr} 
				4 & 1\\
				-1 &1 
			\end{array} 
			\right]
			& b. & B=	\left[
			\begin{array}{rr} 
				-1 & 2\\
				3  & 4 
			\end{array} 
			\right]
			& c. & C=	\left[
			\begin{array}{rrr} 
				1 & 2 & -1\\
				2 & -1 & 3\\
				4 & -2 & 1 
			\end{array} 
			\right]
			& d. &  D=\left[
			\begin{array}{rrr} 
				2 & 4 & 3\\
				3 & 1 & 0\\
				-2 &1 & -2 
			\end{array} 
			\right]
			\\
		\end{array} 
		$$
		\item Selesaikan sistem linier berikut dengan menggunakan transformasi matriks\\ %3
				$$
		\begin{array}{lclc} 
			a. & 
			\begin{array}{rcr} 
				4x_1 + x_2&=& 11 \\
				-x_1 + x_2&=& 1 
			\end{array} ~~~~~~~~~~~~
			& b. & 
			\begin{array}{rcr} 
				-x_1 + 2x_2&=& -1 \\
				3x_1 + 4x_2&=& 13 
			\end{array} \\
		\\
			c. & \begin{array}{rcr} 
				x_1 + 2x_2  - x_3 &=& 4 \\
				2x_1 - x_2  + 3x_3 &=& 3  \\
				4x_1 - 2x_2  + x_3 &=& 1 
			\end{array} ~~~~~~~~~~~~
			& d. &  \begin{array}{lcr} 
				2x_1 + 4x_2  + 3x_3 &=& 13 \\
				3x_1 + x_2  &=& 10  \\
				-2x_1 + x_2  - 2x_3 &=& -7 
			\end{array}
		\end{array}
		$$
		\end{enumerate}
\begin{thebibliography}{2}
    \bibitem{zz}Zaitun,				                   %Authors
        \newblock  (2023).					       % Years
        \newblock TULISAN PRIBADI.			       % Title
        \newblock \emph{Journal Name} number issue(section),  % Jurnal name and volume
        \newblock City;										  % City
        \newblock Page number.

    \bibitem{zz1}Zaitun,				                   %Authors
        \newblock  (2023).					       % Years
        \newblock TULISAN PRIBADI.			       % Title
        \newblock \emph{Journal Name} number issue(section),  % Jurnal name and volume
        \newblock City;										  % City
        \newblock Page number.

    \bibitem{zz2}Zaitun,				                   %Authors
        \newblock  (2023).					       % Years
        \newblock TULISAN PRIBADI.			       % Title
        \newblock \emph{Journal Name} number issue(section),  % Jurnal name and volume
        \newblock City;										  % City
        \newblock Page number.	

    \bibitem{zz3}Zaitun,				                   %Authors
        \newblock  (2023).					       % Years
        \newblock TULISAN PRIBADI.			       % Title
        \newblock \emph{Journal Name} number issue(section),  % Jurnal name and volume
        \newblock City;										  % City
        \newblock Page number.									  % Pages
\end{thebibliography}
\end{document}