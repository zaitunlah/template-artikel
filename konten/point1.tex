		\section{Materi dasar 1}
		\paragraph*{} \lipsum[3]\cite{zz1}
		\begin{align*}
		A = \left[
		\begin{array}{rr} 
			2 & 1 \\
			3 & -1 
		\end{array} 
		\right]
		\end{align*}
		\lipsum[7]
		\subsection[]{Materi dasar 1.1} 
		\lipsum[2]
		$$K_{ij} = (-1)^{i+j}M_{ij}$$
		\begin{align*}
			K_{22} &= (-1)^{4}M_{22}\\
			K_{22} &= 5
		\end{align*}
		\subsection{Materi dasar 1.2}
		% Transformasi ini tidak membatasi ordo matriks, sehingga untuk ordo berapapun dapat diselesaikan. Tambahan bahwa untuk ukuran matriks yang lebih besar, dapat diterapkan dua operasi sekaligus agar langkah proses lebih efisien. Misalkan matriks ordo $3\times3$ berikut
				\begin{align*}
		\left[
		\begin{array}{rrr} 
			-2 & 1 & 2\\
			2 & 3 & 1\\
			-2 &1 & 3 
		\end{array} 
		\right]
	\end{align*}
	\lipsum[6]